
\section{Introducción.}\label{cap:intro}
En el presente documento se realizará un análisis a profundidad de los requerimientos funcionales y no funcionales para la realización del proyecto
Cariocas, el cual consiste en la implementación de un sistema que permita jugar Cariocas de manera digital a tráves de una terminal de computadora.\\
\\Según lo solicitado por el cliente, el sistema debe ser capaz de implementar las reglas del juego de Cariocas,
permitiendo que los jugadores puedan jugar de manera rápida y sin complicaciones, además de otras solicitudes 
que se detallarán a profundidad en este documento.\\
\\Además de lo anterior, está incluido en este análisis, un apartado orientado a diagramas de casos de uso,
diagramas de clases y diagramas de secuencia, los cuales permitirán visualizar de manera más clara la implementación del sistema.\\



\subsection{Próposito del Sistema.}\label{cap:proposito}
Continuando con el próposito del sistema de Cariocas, es importante destacar que el objetivo es 
brindar una experiencia de juego sencilla y sin complicaciones para los dos jugadores. Para lograr esto, 
el sistema debe ser fácil de usar y permitir que los jugadores realicen una partida de manera rápida y sin problemas.\\

Además, el sistema deberá proporcionar un entorno integral que permita a los jugadores competir de manera justa, 
correcta e imparcial. Esto significa que el sistema tendrá que garantizar que la partida se desarrolle de acuerdo a las reglas del juego y sin 
favoritismos ni trucos por parte de ningún jugador.\\

En resumen, el próposito del sistema de Cariocas es proporcionar una experiencia de juego sencilla y justa para los 
jugadores, permitiendo que disfruten de la partida de manera plena y sin complicaciones.



\subsection{Alcance del Sistema.}\label{cap:alcance}
El sistema se va a encargar de implementar la logica y reglas del juego presentes en la web de referencia del presente documento, 
podrá ser jugado por 2 personas y además contará con un registro de puntaje. Mencionado lo anterior, es necesario destacar 
que en todo momento los jugadores pueden observar las cartas de sus contrincantes.

\subsection{Objetivos y criterios de éxito del proyecto.}\label{cap:objetivos}
\begin{itemize}
    \item Garantizar la correcta resolución del juego, libre de errores y contratiempos que afecten la dinámica de este.
    \item Proporcionar un sistema el cual se pueda utilizar mediante una linea de comandos.
    \item Los jugadores puedan jugar las 2 primeras rondas en donde se realizan los patrones de dos trios, y posteriormente, una
    escala y un trio.
    \item El jugador que esté jugando la ronda puede sacar una carta del mazo.
    \item El jugador que esté jugando la ronda puede descartar una carta de su mano.
    \item El jugador que esté jugando la ronda puede bajar el patrón solicitado por la ronda
    \item Finalmente, el jugador puede botar una carta de su mano en patrones del otro jugador.
    \item El jugador puede ganar una partida de cariocas, o perderla según sea el caso.
\end{itemize}

\pagebreak
\subsection{Definiciones, siglas y abreviaturas.}\label{cap:definiciones} 
\begin{itemize}
    \item \textbf{Cariocas}: Juego de mesa para dos jugadores.
    \item \textbf{Jugador}: Persona que participa en una partida de Cariocas.
    \item \textbf{Partida}: Juego de Cariocas entre dos jugadores.
    \item \textbf{Baraja}: Conjunto de cartas que se utilizan en una partida de Cariocas.
    \item \textbf{Ronda}: Una ronda es un conjunto de logros que se deben realizar en una partida de Cariocas. Para el sistema propuesto en este documento, se consideran 2 rondas.
    \item \textbf{Mano}: Conjunto de cartas que posee un jugador en una partida de Cariocas.
    \item \textbf{Mazo}: Conjunto de cartas que se encuentran en la mesa y que pueden ser utilizadas por los jugadores.
    \item \textbf{Patrón}: Conjunto de cartas que se deben realizar en una ronda de Cariocas.
    \item \textbf{Bajada}: Acción de colocar un patrón en la mesa.
    \item \textbf{Botar}: Acción de colocar una carta de la mano en un patrón de otro jugado
    \item \textbf{Descartar}: Acción de colocar una carta de la mano en la baraja de descarte.
    \item \textbf{Sacar}: Acción de colocar una carta de la baraja de descarte o del mazo en la mano.
    \item \textbf{Carta}: Una de las 52 cartas que componen una baraja.
    \item \textbf{Pinta}: Símbolos que agrupan un determinado tipo de carta, entre ellas existen los siguientes grupos: Picas, Corazones, Rombos y Tréboles.
    \item \textbf{Trio}: Agrupación de tres cartas de un mismo valor númerico.
    \item \textbf{Escala}: Secuencia de cuatro cartas de una misma pinta y con valor númerico ascendente.
    \item \textbf{Jack (J)}: Corresponde a la primera carta proveniente después de la carta número diez.
    \item \textbf{Queen (Q)}: Corresponde a la segunda carta proveniente después de la carta número diez.
\end{itemize} 

\subsection{Referencias}
Las referencias utilizadas en este informe provienen de la siguiente documentación:
\begin{itemize}
    \item \textit{Documentación Oficial de Python} \cite{python}
    \item \textit{¿Como jugar a las cariocas?} \cite{cariocas}
\end{itemize}