
\section{Introducción.}\label{cap:intro}

En el presente documento se procedera a realizar un análisis a profundidad de la propuesta de software para 
el determinado juego de mesa: Cariocas. El objetivo es determinar las necesidades del 
cliente, así como también las características del sistema a desarrollar. Para ello, se procederá a realizar 
un análisis de los requerimientos funcionales y no funcionales del sistema, así como también del usuario, entre otros. \\

Siguiendo con el análisis del software para el juego de mesa Cariocas, es importante tener en cuenta que el objetivo es satisfacer las necesidades del cliente y desarrollar un sistema que cumpla con sus expectativas y requerimientos. 
Para ello, se deben considerar tanto los requerimientos funcionales como los no funcionales del sistema.\\

Los requerimientos funcionales se refieren a las tareas y funciones que el sistema debe realizar para cumplir con su propósito. 
Por otro lado, los requerimientos no funcionales se refieren a las características que deben cumplir el sistema para ser adecuado para su uso.\\

Además, es importante realizar un análisis del usuario para entender sus necesidades y expectativas con respecto al sistema. 
Esto incluye considerar aspectos como el perfil del usuario, sus hábitos de uso y su experiencia previa con sistemas similares.\\

En resumen, el análisis del software para el juego de mesa Cariocas debe abarcar tanto los requerimientos funcionales como no funcionales del sistema, así como también un análisis del usuario para garantizar que el sistema cumpla con sus necesidades y expectativas.


\subsection{Próposito del Sistema.}\label{cap:proposito}
Continuando con el próposito del sistema de Cariocas, es importante destacar que el objetivo es 
brindar una experiencia de juego sencilla y sin complicaciones para los dos jugadores. Para lograr esto, 
el sistema debe ser fácil de usar y permitir que los jugadores realicen una partida de manera rápida y sin problemas.\\

Además, el sistema deberá proporcionar un entorno integral que permita a los jugadores competir de manera justa, 
correcta e imparcial. Esto significa que el sistema tendrá que garantizar que la partida se desarrolle de acuerdo a las reglas del juego y sin 
favoritismos ni trucos por parte de ningún jugador.\\

En resumen, el próposito del sistema de Cariocas es proporcionar una experiencia de juego sencilla y justa para los 
jugadores, permitiendo que disfruten de la partida de manera plena y sin complicaciones.

\subsection{Alcance del Sistema.}\label{cap:alcance}
El sistema se va a encargar de implementar la logica y reglas del juego de tal forma que los jugadores puedan
interactuar con este a traves de una terminal de computadora. Mencionado lo anterior, es necesario destacar 
que en todo momento los jugadores pueden observar las cartas de sus contrincantes.

\subsection{Objetivos y criterios de éxito del proyecto.}\label{cap:objetivos}
\begin{itemize}
    \item Entregar un sistema que permita a los jugadores jugar de manera rápida y sin complicaciones.
    \item Garantizar la correcta resolución del juego, libre de errores y contratiempos que afecten la dinámica de este.
    \item Proporcionar un sistema el cual sea fácil e intuitivo de seguir y usar.
    \item Entregar un sistema en el que ningún jugador se vea beneficiado más allá de las reglas del juego.
\end{itemize}


\subsection{Definiciones, siglas y abreviaturas.}\label{cap:definiciones}
\begin{itemize}
    \item \textbf{Cariocas:} Juego de mesa para dos jugadores.
    \item \textbf{Jugador:} Persona que participa en una partida de Cariocas.
    \item \textbf{Partida:} Juego de Cariocas entre dos jugadores.
    \item \textbf{Baraja:} Conjunto de cartas que se utilizan en una partida de Cariocas.
    \item \textbf{Carta:} Una de las 52 cartas que componen una baraja.
    \item \textbf{Pinta:} Símbolos que agrupan un determinado tipo de carta, entre ellas existen los siguientes grupos: Picas, Corazones, Rombos y Tréboles.
    \item \textbf{Trio:} Agrupación de tres cartas de un mismo valor númerico.
    \item \textbf{Escala:} Secuencia de cuatro cartas de una misma pinta y con valor númerico ascendente.
    \item \textbf{Escala Real:} Secuencia de doce cartas de una misma pinta y un valor númerico ascendente, corresponden a todas las cartas de la misma pinta exceptuando las cartas Joker.
    \item \textbf{Jack (J):} Corresponde a la primera carta proveniente después de la carta número diez.
    \item \textbf{Queen (Q):} Corresponde a la segunda carta proveniente después de la carta número diez.
\end{itemize}
\subsection{Referencias}
Las referencias utilizadas en este informe provienen de los siguientes libros:
\begin{itemize}
    \item \textit{Ingeniería de software orientado a objetos. Bruegge,Bernd y Dutoit,Allen H.} \cite{oop}
\end{itemize}