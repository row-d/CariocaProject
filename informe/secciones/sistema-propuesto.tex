\pagebreak

\section{Sistema propuesto.}\label{cap:sistema}
\subsection{Panorama.}\label{cap:panorama}

El sistema permitirá al usuario realizar las operaciones básicas del juego de una forma eficaz y libre de errores, 
las cuales consisten en acciones relacionadas al retirar carta de la baraja, bajar mano, entre otras relacionadas. 
El sistema se deberá se desarrollar en Python 3 por lo que contará con una interfaz en base a linea de comandos.
También contará con un sistema de puntuación que permitirá al usuario saber su puntaje en cada ronda y en el juego en general.

\subsection{Requerimientos Funcionales.}\label{cap:requerimientos-funcionales}

Es necesario que pueda ser jugado por al menos dos jugadores. Para ello, se propone la implementación de un menú de interfaz de línea de comandos, el cual permitirá a los jugadores realizar diferentes acciones dentro del juego.

El juego comenzará barajeando las cartas y repartiendo 12 cartas a cada jugador. Cada jugador deberá analizar sus cartas y decidir qué acción realizará en su turno. Las acciones disponibles son las siguientes:
\begin{itemize}
  \item Tomar una carta de la baraja.
  \item Descartar una carta de su mano.
  \item Bajar una carta de su mano.
  \item Botar una carta de su mano.
  \item Ver las cartas de la baraja.
  \item Ver las cartas de la mesa.
  \item Ver las cartas de la pila de descartes.
  \item Ver las cartas de la pila de bajas.
  \item Ver las cartas de la pila de bajas de los demás jugadores.
\end{itemize}

El jugador debera cumplir con el contrato de cada ronda, 
el cual consiste en reunir un determinado patrón de cartas basado en su valor númerico o en sus pintas, tales como trios o escalas.


\subsection{Requerimientos No Funcionales.}\label{cap:requerimientos-no-funcionales}

Para que el juego de cariocas pueda ofrecer una experiencia completa y
satisfactoria a los usuarios, es necesario que se implemente una persistencia 
temporal de datos para almacenar puntuaciones del juego mientras el programa esta en ejecución. 
El juego contará con una interfaz en base a linea de comandos.


\subsection{Seudorrequerimientos.}\label{cap:seudorrequerimientos}
Por solicitud del cliente, es necesario que el software sea desarrollado únicamente utilizando Python 3 como lenguaje de programación. Además, se requiere que toda la documentación del proyecto sea escrita en LaTeX.
