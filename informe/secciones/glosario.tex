\section{Glosario}\label{cap:glosario}
\begin{itemize}
        \item \textbf{Cariocas}: Juego de mesa para dos jugadores.
        \item \textbf{Jugador}: Persona que participa en una partida de Cariocas.
        \item \textbf{Partida}: Juego de Cariocas entre dos jugadores.
        \item \textbf{Ronda}: Una ronda es un conjunto de logros que se deben realizar en una partida de Cariocas. Para el sistema propuesto en este documento, se consideran 2 rondas.
        \item \textbf{Mano}: Conjunto de cartas que posee un jugador en una partida de Cariocas.
        \item \textbf{Mazo}: Conjunto de cartas que se encuentran en la mesa y que pueden ser utilizadas por los jugadores.
        \item \textbf{Patrón}: Conjunto de cartas que se deben realizar en una ronda de Cariocas.
        \item \textbf{Bajada}: Acción de colocar un patrón en la mesa.
        \item \textbf{Botar}: Acción de colocar una carta de la mano en un patrón de otro jugado
        \item \textbf{Descartar}: Acción de colocar una carta de la mano en la baraja de descarte.
        \item \textbf{Sacar}: Acción de colocar una carta de la baraja de descarte o del mazo en la mano.
        \item \textbf{Carta}: Una de las 52 cartas que componen una baraja.
        \item \textbf{Pinta}: Símbolos que agrupan un determinado tipo de carta, entre ellas existen los siguientes grupos: Picas, Corazones, Rombos y Tréboles.
        \item \textbf{Trio}: Agrupación de tres cartas de un mismo valor númerico.
        \item \textbf{Escala}: Secuencia de cuatro cartas de una misma pinta y con valor númerico ascendente.
        \item \textbf{Jack (J)}: Corresponde a la primera carta proveniente después de la carta número diez.
        \item \textbf{Queen (Q)}: Corresponde a la segunda carta proveniente después de la carta número diez.
\end{itemize}
\clearpage

\medskip
\printbibliography

